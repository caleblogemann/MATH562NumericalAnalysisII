\documentclass[11pt]{article}
\usepackage[letterpaper]{geometry}
\usepackage{MATH562}

\begin{document}
\noindent \textbf{\Large{Caleb Logemann \\
MATH 562 Numerical Analysis II \\
Homework 2
}}

%\lstinputlisting[language=Matlab]{H01_23.m}
\begin{enumerate}
    \item % #1 Problem 6.1 page 47
        Let $P$ be an orthogonal projector.
        \begin{enumerate}
            \item[(a)] % Done
                Prove that $I - 2P$ is unitary.

                \begin{proof}
                    Let $P$ be an orthogonal projector, that is $P^2 = P$ and
                    $P = P^*$.
                    Consider $(I - 2P)^* (I - 2P)$.
                    \begin{align*}
                        (I - 2P)^* (I - 2P) &= (I - 2P^*) (I - 2P) \\
                                            &= I - 2P - 2P^* + 4P^*P
                        \intertext{Since $P^* = P$}
                        I - 2P - 2P^* + 4P^*P &= I - 4P + 4P^2
                        \intertext{Since $P^2 = P$.}
                        I - 4P + 4P^2 &= I
                    \end{align*}
                    Therefore since $(I - 2P)^* (I - 2P) = I$, $I - 2P$ is unitary.
                \end{proof}

            \item[(b)]
                Describe the action of $I - 2P$ geometrically.
                % What can you say about the relationship between a point and
                % its image


        \end{enumerate}

    \item % #2 Problem 6.3 page 47 Done
        Suppose that $A \in \CC^{m \times n}$ with $m \ge n$.
        \begin{enumerate}
            \item[(a)] % Done
                Show that $A^* A$ is nonsingular if and only if $A$ has full
                rank.

                \begin{proof}
                    Let $A \in \CC^{m \times n}$ with $m \ge n$.
                    Suppose that $A^* A$ is nonsingular, this implies that
                    $\nullspace{A^* A} = \set{\v{0}}$.
                    Now consider $\v{x} \in \nullspace{A}$, that is
                    $A\v{x} = \v{0}$.
                    By right multiplying by $A^*$, we conclude that
                    $A^* A \v{x} = \v{0}$.
                    This implies that $\v{x} \in \nullspace{A^* A}$ and
                    therefore $\v{x} = \v{0}$.
                    This implies that $\nullspace{A} = \set{\v{0}}$, that is
                    $A$ has full rank.

                    Now suppose that $A$ full rank, that is
                    $\nullspace{A} = \v{0}$.
                    Now consider $\v{x} \in \nullspace{A^* A}$, that is
                    $A^* A \v{x} = \v{0}$.
                    By right multiplying by $\v{x}^*$, we note that
                    \begin{align*}
                        \v{x}^* A^* A \v{x} &= 0 \\
                        \p{A \v{x}}^* A\v{x} &= 0 \\
                        \norm[2]{A\v{x}}^2 &= 0.
                    \end{align*}
                    The only way the for the 2-norm of a vector to be zero
                    is if that vector is the zero vector, therefore
                    $A\v{x} = \v{0}$.
                    Therefore $\v{x} \in \nullspace{A}$ and $\v{x} = \v{0}$.
                    This implies that $\nullspace{A^* A} = \set{\v{0}}$, which
                    implies that $A^* A$ is nonsingular.
                \end{proof}

            \item[(b)] % Done
                Show that if $A$ has full rank, then $P = A\p{A^* A}^{-1} A^*$
                is an orthogonal projector onto the range of $A$.

                \begin{proof}
                    Let $A \in \CC^{m \times n}$ with $m \ge n$ have full rank.
                    Define $P = A\p{A^* A}^{-1} A^*$.
                    Consider $P^2$.
                    \begin{align*}
                        P^2 &= A\p{A^* A}^{-1} A^* A\p{A^* A}^{-1} A^* \\
                            &= A\p{A^* A}^{-1} \p{A^* A} \p{A^* A}^{-1} A^* \\
                            &= A \p{A^* A}^{-1} A^* \\
                            &= P
                    \end{align*}
                    Therefore $P$ is a projector.
                    Now consider $P^*$
                    \begin{align*}
                        P^* &= \p{A \p{A^* A}^{-1} A^*}^* \\
                            &= A \p{A \p{A^* A}^{-1}}^* \\
                            &= A \p{\p{A^* A}^{-1}}^* A^* \\
                    \end{align*}
                    Notice that $A^* A$ is Hermitian, and that it is known that
                    the inverse of a Hermitian matrix is also Hermitian.
                    This implies that $\p{\p{A^* A}^{-1}}^* = \p{A^* A}^{-1}$, so
                    \begin{align*}
                        P^* &= A \p{\p{A^* A}^{-1}}^* A^* \\
                            &= A \p{A^* A}^{-1} A^* \\
                            &= P
                    \end{align*}
                    We can now conclude that $P$ is an orthogonal projector.
                    Finally we need to show that $P$ projects onto the range of
                    $A$, that is for every vector,  $\v{x} \in \range(A)$,
                    there exists a vector $\v{y}$ such that $P\v{y} = \v{x}$.
                    Let $\v{x} \in \range(A)$, then there exists $\v{z}$ such that
                    $A\v{z} = \v{x}$.
                    Since $(A^* A)$ is nonsingular $(A^* A)^{-1}$ exists.
                    Therefore $A\v{z} = A(A^* A)^{-1}(A^* A)\v{z}$.
                    This can be rewritten as $A(A^* A)^{-1}A^* \v{x} = P\v{x}$.
                    Therefore $P\v{x} = \v{x}$ and $\v{x} \in \range(P)$.
                    Therefore $P$ projects onto the range of $A$.
                \end{proof}
        \end{enumerate}

    \item % #3 Problem 7.5 page 55
        Suppose that $A \in \CC^{m \times n}$ with $m \ge n$, and let
        $A = \hat{Q}\hat{R}$ be the reduced $QR$ factorization of $A$.
        \begin{enumerate}
            \item[(a)]
                Show that $A$ has full rank if and only if all the diagonal
                entries of $\hat{R}$ are nonzero.

            \item[(b)]
                Suppose that $\hat{R}$ has $k$ nonzero diagonal entries and
                $n - k$ zero diagonal entries.
                What does that imply about the rank of $A$.
                Justify your answer.
        \end{enumerate}

    \item % #4 Problem 8.1 page 61 Done
        Let $A$ be an $(m \times n)$ matrix.
        Determine the exact number of floating point additions, subtractions,
        mutliplications, and divisions involved in computing the reduced $QR$
        factorization of $A$ using algorithm 8.1 on page 58.

        The first for loop in algorithm 8.1 contains no floating point
        operations, and thus does not contribute to the overall flop count.
        The second for loop does contain floating point operations.
        This for loop has an inner nested loop, I will first count the
        number of operations in this inner loop.
        The first operation in this inner loop is a multiplication of a row
        vector and column vector each of length $m$.
        Therefore this operation includes $m$ multiplications and $m-1$
        additions.
        The second operation $\v{v}_j - r_{ij} \v{q}_i$ is the multiplication
        of one vector by a constant, $m$ operations, and the subtraction of
        two vectors, another $m$ operations.
        Therefore each iteration of this loop includes
        $m + m - 1 + m + m = 4m - 1$ operations.
        This loop iterates from $j = i + 1$ to $n$, so the total number of
        operations for this loop can be described as the following sum.
        \begin{align*}
            \sum{j = i+1}{n}{4m - 1} &= \p{4m - 1}{n - i}
        \end{align*}
        The outer for loop first compute $\norm{\v{v}_i}$, assuming this is
        the two norm then this will take $m$ multiplications, $m - 1$ additions,
        and $1$ squareroot operation.
        Therefore the entire norm computations requires $2m$ flops.
        Even if this does not refer to the 2-norm, the number of flops will be
        similar.
        The next computation divides a vector by a constant, this requires $m$
        divisions.
        Therefore the one iteration of the outer for loop requires $3m$ flops
        and the number of flops in the inner loop.
        This loop iterates from $i = 1$ to $n$, so the total flop count can be
        expressed as follows.
        \begin{align*}
            \sum{i = 1}{n}{3m + \p{4m - 1}\p{n-i}} &= \sum{i = 1}{n}{3m} + \p{4m-1}\sum{i=1}{n}{n-i} \\
            &= 3mn + \p{4m - 1}\p{\sum{i=1}{n}{n} - \sum{i=1}{n}{i}} \\
            &= 3mn + \p{4m - 1}\p{n^2 - \frac{n(n+1)}{2}} \\
            &= 3mn + \p{4m - 1}\p{\frac{n^2-n}{2}} \\
            &= 3mn + \p{4m - 1}\p{\frac{n^2-n}{2}} \\
            &= 2mn^2 + mn - \frac{1}{2}n^2 + \frac{1}{2}n
        \end{align*}
        Thus the total number of flops for algorithm 8.1 is
        $2mn^2 + mn - \frac{1}{2}n^2 + \frac{1}{2}n$.

    \item % #5 Problem 10.1 page 76
        Let $F = I - 2\v{u}\v{u}^T$ be a Householder reflector on $\RR^m$.
        Deterimine the eigenvalues, the determinant, and the singular values of
        $F$.
        Give a geometric argument supporting your algebraic eigenvalue
        computation.

    \item % #6 Done
        \begin{enumerate}
            \item[(a)] % Done
                Let $\v{x}, \v{y} \in \CC^m$ satisfy i) $\v{x} \neq \v{y}$, ii)
                $\norm[2]{\v{x}} = \norm[2]{\v{y}}$, iii)
                $\v{x}^* \v{y} \in \RR$.
                Show that there is a reflector $F$ (satisfying $F^* = F$ and $F^2 = I$)
                such that $F\v{x} = \v{y}$.

                Let $F = I - 2 \frac{(\v{y} - \v{x})(\v{y} - \v{x})^*}{(\v{y} - \v{x})^*(\v{y} - \v{x})}$.
                Since $\v{x}^* \v{y} \in \RR$, then
                $\bar{\v{x}^* \v{y}} = \v{y}^* \v{x} \in \RR$ and
                $\v{x}^* \v{y} = \v{x}^* \v{y}$.
                This allows for $F$ to be simplified as follows
                \begin{align*}
                    F &= I - 2 \frac{(\v{y} - \v{x})(\v{y} - \v{x})^*}{(\v{y} - \v{x})^*(\v{y} - \v{x})} \\
                      &= I - 2 \frac{(\v{y} - \v{x})(\v{y} - \v{x})^*}{\v{y}^*\v{y} - \v{y}^*\v{x} - \v{x}^*\v{y} + \v{x}^*\v{x}} \\
                      &= I - 2 \frac{(\v{y} - \v{x})(\v{y} - \v{x})^*}{2\norm[2]{\v{x}}^2 - 2\v{x}^*\v{y}} \\
                      &= I - \frac{(\v{y} - \v{x})(\v{y} - \v{x})^*}{\norm[2]{\v{x}}^2 - \v{x}^*\v{y}} \\
                \end{align*}
                Now consider $F^*$
                \begin{align*}
                    F^* &= \p{I - \frac{(\v{y} - \v{x})(\v{y} - \v{x})^*}{\norm[2]{\v{x}}^2 - \v{x}^*\v{y}}}^* \\
                        &= I^* - \frac{\p{(\v{y} - \v{x})(\v{y} - \v{x})^*}^*}{\norm[2]{\v{x}}^2 - \v{x}^*\v{y}} \\
                        &= I - \frac{(\v{y} - \v{x})(\v{y} - \v{x})^*}{\norm[2]{\v{x}}^2 - \v{x}^*\v{y}} \\
                        &= F
                \end{align*}
                Also consider $F^2$
                \begin{align*}
                    F^2 &= \p{I - \frac{(\v{y} - \v{x})(\v{y} - \v{x})^*}{\norm[2]{\v{x}}^2 - \v{x}^*\v{y}}}^2 \\
                        &= I^2 - 2\frac{(\v{y} - \v{x})(\v{y} - \v{x})^*}{\norm[2]{\v{x}}^2 - \v{x}^*\v{y}} + \p{\frac{(\v{y} - \v{x})(\v{y} - \v{x})^*}{\norm[2]{\v{x}}^2 - \v{x}^*\v{y}}}^2 \\
                        &= I - 2\frac{(\v{y} - \v{x})(\v{y} - \v{x})^*}{\norm[2]{\v{x}}^2 - \v{x}^*\v{y}} + \frac{(\v{y} - \v{x})(\v{y} - \v{x})^*(\v{y} - \v{x})(\v{y} - \v{x})^*}{\p{\norm[2]{\v{x}}^2 - \v{x}^*\v{y}}^2} \\
                        &= I - 2\frac{(\v{y} - \v{x})(\v{y} - \v{x})^*}{\norm[2]{\v{x}}^2 - \v{x}^*\v{y}} + 2\p{\norm[2]{\v{x}}^2 - \v{x}^*\v{y}}\frac{(\v{y} - \v{x})(\v{y} - \v{x})^*}{\p{\norm[2]{\v{x}}^2 - \v{x}^*\v{y}}^2} \\
                        &= I - 2\frac{(\v{y} - \v{x})(\v{y} - \v{x})^*}{\norm[2]{\v{x}}^2 - \v{x}^*\v{y}} + 2\frac{(\v{y} - \v{x})(\v{y} - \v{x})^*}{\norm[2]{\v{x}}^2 - \v{x}^*\v{y}} \\
                        &= I
                \end{align*}
                Thus $F$ is a reflector.
                Finally we can examine $F\v{x}$.
                \begin{align*}
                    F\v{x} &= \p{I - \frac{(\v{y} - \v{x})(\v{y} - \v{x})^*}{\norm[2]{\v{x}}^2 - \v{x}^*\v{y}}}\v{x} \\
                           &= \v{x} - (\v{y} - \v{x})\frac{(\v{y} - \v{x})^*\v{x}}{\norm[2]{\v{x}}^2 - \v{x}^*\v{y}} \\
                           &= \v{x} - (\v{y} - \v{x})\frac{\v{y}^* \v{x} - \v{x}^*\v{x}}{\norm[2]{\v{x}}^2 - \v{x}^*\v{y}} \\
                           &= \v{x} + (\v{y} - \v{x})\frac{\norm[2]{\v{x}}^2 - \v{x}^* \v{y}}{\norm[2]{\v{x}}^2 - \v{x}^*\v{y}} \\
                           &= \v{x} + \v{y} - \v{x} \\
                           &=  \v{y}
                \end{align*}

            \item[(b)] % Done
                Let $\v{x} \in \CC^m$, $\v{x} \neq 0$.
                The polar form of the first component of $\v{x}$ is
                $x_1 = re^{i\theta}$.
                Set $\v{y} = \norm[2]{\v{x}} e^{i\theta} \v{e}_1$.
                Assuming $\v{x}$ is not a mutliple of $\v{e}_1$, show that
                $\v{x}, \v{y}$ satisfy properties i), ii), and iii) above.

                Since $\v{x}$ is not a multiple of $\v{e}_1$ there exists an
                index $i \neq 1$ such that $x_i \neq 0$.
                However $y_i = 0$ for all $i \neq 1$, therefore $\v{x} \neq \v{y}$
                and property i) is satisfied.
                Consider $\norm[2]{\v{y}}$
                \begin{align*}
                    \norm[2]{\v{y}} &= \norm[2]{\norm[2]{\v{x}} e^{i\theta} \v{e}_1} \\
                    &= \norm[2]{\v{x}} \abs{e^{i\theta}} \norm[2]{\v{e}_1} \\
                    &= \norm[2]{\v{x}} \times 1 \times 1 \\
                    &= \norm[2]{\v{x}}.
                \end{align*}
                Therefore property ii) is satisfied.
                Now consider $\v{x}^* \v{y}$.
                Since $\v{y}$ is a multiple $\v{e}_1$,
                \begin{align*}
                    \v{x}^* \v{y} &= \bar{x_1} \times y_1 \\
                    &= \bar{re^{i\theta}} \norm[2]{\v{x}} e^{i\theta} \\
                    &= r e^{-i\theta} \norm[2]{\v{x}} e^{i\theta} \\
                    &= r \norm[2]{\v{x}} e^0 \\
                    &= r \norm[2]{\v{x}} \in \RR
                \end{align*}
                Therefore property iii) is satisfied.
        \end{enumerate}

    \item % #7 Done
        Write a MATLAB function [$W$,$R$] = house($A$) that takes as input a
        $(m\times n)$ matrix $A$ and returns an implicit representation of the
        full $QR$ factorization of $A$.
        The matrix $W$ should be the lower triangular matrix whose columns are
        the vectors $\v{v}_1,\ldots,\v{v}_n$ where
        $\v{v}_k = \norm[2]{\v{x}_k}\v{e}_1 − \v{x}_k$ is used to define the
        Householder reflector $F_k$ at the k-th stage of the process.
        $R$ should be the triangular factor in the factorization.
        Some test matrices will be supplied next week.
        You should turn in your code and the output from the test matrices for
        this problem.

        My function and the code to test the two test matrices are shown below.
        \lstinputlisting[language=Matlab]{HouseholderQR.m}
        \lstinputlisting[language=Matlab]{H02.m}
        This code gives the following output
        \begin{verbatim}
                >> H02

                W =

                    0.7420         0         0
                    0.2723    0.7866         0
                    0.4765    0.1192    0.1988
                    0.2723   -0.4284    0.9081
                    0.2723   -0.4284   -0.3685


                R =

                -9.8995   -9.4954   -9.6975
                -0.0000   -3.2919   -3.0129
                -0.0000   -0.0000   -1.9701
                -0.0000    0.0000         0
                -0.0000    0.0000         0


                W =

                    0.8660         0
                -0.2887    0.8165
                -0.2887   -0.4082
                -0.2887   -0.4082


                R =

                -1.0000   -5.0000
                -0.0000  -10.0000
                -0.0000    0.0000
                -0.0000    0.0000
        \end{verbatim}

\end{enumerate}
\end{document}
