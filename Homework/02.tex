\documentclass[11pt]{article}
\usepackage[letterpaper]{geometry}
\usepackage{MATH562}

\begin{document}
\noindent \textbf{\Large{Caleb Logemann \\
MATH 562 Numerical Analysis II \\
Homework 2
}}

%\lstinputlisting[language=Matlab]{H01_23.m}
\begin{enumerate}
    \item % #1 Problem 6.1 page 47
        Let $P$ be an orthogonal projector.
        \begin{enumerate}
            \item[(a)] % Done
                Prove that $I - 2P$ is unitary.

                \begin{proof}
                    Let $P$ be an orthogonal projector, that is $P^2 = P$ and
                    $P = P^*$.
                    Consider $(I - 2P)^* (I - 2P)$.
                    \begin{align*}
                        (I - 2P)^* (I - 2P) &= (I - 2P^*) (I - 2P) \\
                                            &= I - 2P - 2P^* + 4P^*P
                        \intertext{Since $P^* = P$}
                        I - 2P - 2P^* + 4P^*P &= I - 4P + 4P^2
                        \intertext{Since $P^2 = P$.}
                        I - 4P + 4P^2 &= I
                    \end{align*}
                    Therefore since $(I - 2P)^* (I - 2P) = I$, $I - 2P$ is unitary.
                \end{proof}

            \item[(b)]
                Describe the action of $I - 2P$ geometrically.
                % What can you say about the relationship between a point and
                % its image

                
        \end{enumerate}

    \item % #2 Problem 6.3 page 47
        Suppose that $A \in \CC^{m \times n}$ with $m \ge n$.
        \begin{enumerate}
            \item[(a)]
                Show that $A^* A$ is nonsingular if and only if $A$ has full
                rank.

                \begin{proof}
                    Let $A \in \CC^{m \times n}$ with $m \ge n$.
                    Suppose that $A^* A$ is nonsingular, that is
                    $(A^* A)^{-1}$ exists.
                \end{proof}

            \item[(b)] % Done
                Show that if $A$ has full rank, then $P = A\p{A^* A}^{-1} A^*$
                is an orthogonal projector onto the range of $A$.

                \begin{proof}
                    Let $A \in \CC^{m \times n}$ with $m \ge n$ have full rank.
                    Define $P = A\p{A^* A}^{-1} A^*$.
                    Consider $P^2$.
                    \begin{align*}
                        P^2 &= A\p{A^* A}^{-1} A^* A\p{A^* A}^{-1} A^* \\
                            &= A\p{A^* A}^{-1} \p{A^* A} \p{A^* A}^{-1} A^* \\
                            &= A \p{A^* A}^{-1} A^* \\
                            &= P
                    \end{align*}
                    Therefore $P$ is a projector.
                    Now consider $P^*$
                    \begin{align*}
                        P^* &= \p{A \p{A^* A}^{-1} A^*}^* \\
                            &= A \p{A \p{A^* A}^{-1}}^* \\
                            &= A \p{\p{A^* A}^{-1}}^* A^* \\
                    \end{align*}
                    Notice that $A^* A$ is Hermitian, and that it is known that
                    the inverse of a Hermitian matrix is also Hermitian.
                    This implies that $\p{\p{A^* A}^{-1}}^* = \p{A^* A}^{-1}$, so
                    \begin{align*}
                        P^* &= A \p{\p{A^* A}^{-1}}^* A^* \\
                            &= A \p{A^* A}^{-1} A^* \\
                            &= P
                    \end{align*}
                    We can now conclude that $P$ is an orthogonal projector.
                \end{proof}
        \end{enumerate}

    \item % #3 Problem 7.5 page 55
        Suppose that $A \in \CC^{m \times n}$ with $m \ge n$, and let
        $A = \hat{Q}\hat{R}$ be the reduced $QR$ factorization of $A$.
        \begin{enumerate}
            \item[(a)]
                Show that $A$ has full rank if and only if all the diagonal
                entries of $\hat{R}$ are nonzero.

            \item[(b)]
                Suppose that $\hat{R}$ has $k$ nonzero diagonal entries and
                $n - k$ zero diagonal entries.
                What does that imply about the rank of $A$.
                Justify your answer.
        \end{enumerate}

    \item % #4 Problem 8.1 page 61
        Let $A$ be an $(m \times n)$ matrix.
        Determine the exact number of floating point additions, subtractions,
        mutliplications, and divisions involved in computing the reduced $QR$
        factorization of $A$ using algorithm 8.1 on page 58.

    \item % #5 Problem 10.1 page 76
        Let $F = I - 2\v{u}\v{u}^T$ be a Householder reflector on $\RR^m$.
        Deterimine the eigenvalues, the determinant, and the singular values of
        $F$.
        Give a geometric argument supporting your algebraic eigenvalue
        computation.

    \item % #6
        \begin{enumerate}
            \item[(a)]
                Let $\v{x}, \v{y} \in \CC^m$ satisfy i) $\v{x} \neq \v{y}$, ii)
                $\norm[2]{\v{x}} = \norm[2]{\v{y}}$, iii)
                $\v{x}^* \v{y} \in \RR$.
                Show that there is a reflector $F$ (satisfying $F^* = F$ and $F^2 = I$)
                such that $F\v{x} = \v{y}$.

            \item[(b)]
                Let $\v{x} \in \CC^m$, $\v{x} \neq 0$.
                The polar form of the first component of $\v{x}$ is
                $x_1 = re^{i\theta}$.
                Set $\v{y} = \norm[2]{\v{x}} e^{i\theta} \v{e}_1$.
                Assuming $\v{x}$ is not a mutliple of $\v{e}_1$, show that
                $\v{x}, \v{y}$ satisfy properties i), ii), and iii) above.
        \end{enumerate}

    \item % #7
        Write a MATLAB function [$W$,$R$] = house($A$) that takes as input a
        $(m\times n)$ matrix $A$ and returns an implicit representation of the
        full $QR$ factorization of $A$.
        The matrix $W$ should be the lower triangular matrix whose columns are
        the vectors $\v{v}_1,\ldots,\v{v}_n$ where
        $\v{v}_k = \norm[2]{\v{x}_k}\v{e}_1 − \v{x}_k$ is used to define the
        Householder reflector $F_k$ at the k-th stage of the process.
        $R$ should be the triangular factor in the factorization.
        Some test matrices will be supplied next week.
        You should turn in your code and the output from the test matrices for
        this problem.
\end{enumerate}
\end{document}
