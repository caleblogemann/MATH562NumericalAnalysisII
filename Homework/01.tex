\documentclass[11pt]{article}
\usepackage[letterpaper]{geometry}
\usepackage{MATH562}

\begin{document}
\noindent \textbf{\Large{Caleb Logemann \\
MATH 562 Numerical Analysis II \\
Homework 1
}}

\begin{enumerate}
    \item % #1
        Problem 1.1
        Let $B$ be a $4 \times 4$ matrix to which the following operations are
        applied in the given order.
        \begin{enumerate}
            \item[1.] double column 1
            \item[2.] halve row 3
            \item[3.] add row 3 to row 1
            \item[4.] interchange columns 1 and 4
            \item[5.] subtract row 2 from each other rows
            \item[6.] replace column 4 by column 3
            \item[7.] delete column 1
        \end{enumerate}
        The result can be written as a product of 8 matrices one of which is $B$.
        \begin{enumerate}
            \item[(a)]
                What are the other 7 matrices and what order do they appear in
                the matrix?

                The matrix that doubles column 1 is
                \begin{align*}
                    C &=
                    \begin{bmatrix}
                        2 & 0 & 0 & 0 \\
                        0 & 1 & 0 & 0 \\
                        0 & 0 & 1 & 0 \\
                        0 & 0 & 0 & 1
                    \end{bmatrix}
                    \intertext{when right multiplied.
                        The following matrix halves row 3 when left multiplied.}
                    D &=
                    \begin{bmatrix}
                        1 & 0 & 0 & 0 \\
                        0 & 1 & 0 & 0 \\
                        0 & 0 & .5 & 0 \\
                        0 & 0 & 0 & 1
                    \end{bmatrix}
                    \intertext{The following matrix adds row 3 to the row 1 when left multiplied.}
                    E &= 
                    \begin{bmatrix}
                        1 & 0 & 1 & 0 \\
                        0 & 1 & 0 & 0 \\
                        0 & 0 & 1 & 0 \\
                        0 & 0 & 0 & 1
                    \end{bmatrix}
                    \intertext{The following matrix interchanges columns 1 and 4 when right multiplied.}
                    F &=
                    \begin{bmatrix}
                        0 & 0 & 0 & 1 \\
                        0 & 1 & 0 & 0 \\
                        0 & 0 & 1 & 0 \\
                        1 & 0 & 0 & 0
                    \end{bmatrix}
                    \intertext{The following matrix subtracts row 2 from every other row, when left multiplied.}
                    G &=
                    \begin{bmatrix}
                        1 & -1 & 0 & 0 \\
                        0 & 1 & 0 & 0 \\
                        0 & -1 & 1 & 0 \\
                        0 & -1 & 0 & 1
                    \end{bmatrix}
                    \intertext{The following matrix replaces column 4 with column 3 when right multiplied.}
                    H &=
                    \begin{bmatrix}
                        1 & 0 & 0 & 0 \\
                        0 & 1 & 0 & 0 \\
                        0 & 0 & 1 & 1 \\
                        0 & 0 & 0 & 0
                    \end{bmatrix}
                    \intertext{The following matrix deletes column 1 when right multiplied.}
                    I &=
                    \begin{bmatrix}
                        0 & 0 & 0 \\
                        1 & 0 & 0 \\
                        0 & 1 & 0 \\
                        0 & 0 & 1
                    \end{bmatrix}
                \end{align*}
                The resulting matrix product is given by $GEDBCFHI$, where the
                matrices are given above.

            \item[(b)]
                The result can also be written as a product $ABC$ what are $A$
                and $C$?

                In this case $A$ and $C$ are given by the product of the
                matrices to the left and the right of $B$ in the part (a).
                Therefore
                \begin{align*}
                    A &=
                    \begin{bmatrix}
                        1 & -1 & 0.5 & 0 \\
                        0 &  1 & 0 & 0 \\
                        0 & -1 & 0.5 & 0 \\
                        0 & -1 & 0 & 1
                    \end{bmatrix} \\
                    C &=
                    \begin{bmatrix}
                        0 & 0 & 0 \\
                        1 & 0 & 0 \\
                        0 & 1 & 1 \\
                        0 & 0 & 0
                    \end{bmatrix}
                \end{align*}

        \end{enumerate}

    \item % #2

    \item % #3
    \item % #4
    \item % #5
    \item % #6
    \item % #7
\end{enumerate}
\end{document}
