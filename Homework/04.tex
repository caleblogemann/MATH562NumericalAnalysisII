\documentclass[11pt]{article}
\usepackage[letterpaper]{geometry}
\usepackage{MATH562}

\begin{document}
\noindent \textbf{\Large{Caleb Logemann \\
MATH 562 Numerical Analysis II \\
Homework 4
}}

%\lstinputlisting[language=Matlab]{H01_23.m}
\begin{enumerate}
    \item % #1
        For each of the following, show that the statement is correct, or give
        a counter-example. 
        If nothing else is written, assume that $A \in \CC^{m \times m}$.
        \begin{enumerate}
            \item % Done
                If $\lambda$ is an eigenvalue of $A$ and $\mu \in \CC$, then
                $\lambda - \mu$ is an eigenvalue of $A - \mu I$.

                Yes this is a true statement.
                \begin{proof}
                    Let $\v{x}$ be the eigenvector for the eigenvalue $\lambda$,
                    that is $A\v{x} = \lambda \v{x}$.
                    Thus
                    \begin{align*}
                        (A - \mu I)\v{x} &= A\v{x} - \mu I \v{x} \\
                                         &= \lambda \v{x} - \mu \v{x} \\
                                         &= (\lambda - \mu) \v{x}
                    \end{align*}
                    Therefore $\v{x}$ is an eigenvector of $A - \mu I$ and the
                    corresponding eigenvalue is $\lambda - \mu$.
                \end{proof}

            \item % Done
                If $A$ is real and $\lambda$ is an eigenvalue of $A$, then
                $-\lambda$ is an eigenvalue of $A$.

                This is false.
                Consider the matrix
                \[
                    A =
                    \begin{bmatrix}
                        2 & 0 \\
                        0 & 3 \\
                    \end{bmatrix}.
                \]
                The eigenvalues of this matrix are $2$ and $3$, neither $-2$
                nor $-3$ are eigenvalues.

            \item
                If $A$ is real and $\lambda$ is an eigenvalue of $A$, then
                $\bar{\lambda}$ is an eigenvalue of $A$.

            \item
                If $\lambda$ is an eigenvalue of $A$ and $A$ is nonsingular, then
                $\lambda^{-1}$ is an eigenvalue of $A^{-1}$.

            \item % Done
                If all the eigenvalues of $A$ are zero, than $A = 0$.

                This is false.
                Consider the matrix
                \[
                    A =
                    \begin{bmatrix}
                        0 & 1 \\
                        0 & 0
                    \end{bmatrix}.
                \]
                Both of the eigenvalues of this matrix are zero, however
                $A \neq 0$.

            \item
                If $A$ is Hermitian and $\lambda$ is an eigenvalue of $A$

            \item
                If $A$ is diagonalizable and all eigenvalues are equal, then $A$
                is diagonal.
        \end{enumerate}

    \item % #2
        \begin{enumerate}
            \item[(a)]
                Let $A \in \CC^{m \times m}$ be tridiagonal and Hermitian, with
                all of its subdiagonal and superdiagonal entries nonzero.
                Prove that the eigenvalues of $A$ are distinct.

                % Hint show that for any $\lambda \in \CC$, $A - \lambda I$ has
                % rank at least m - 1.

            \item[(b)]
                Let $A$ be upper-Hessenberg, with all of its subdiagonal entries
                nonzero.
                Give an example that shows that the eigenvalues of $A$ are
                not necessarily distinct.
        \end{enumerate}

    \item % #3
        Suppose $A$ is $m \times m$ and has a complete set of orthonormal
        eigenvectors, $\v{q}_1, \ldots, \v{q}_m$, and with corresponding
        eigenvalues $\lambda_1, \ldots, \lambda_m$.
        Assume that the ordering is such that
        $\abs{\lambda_j} \ge \abs{\lambda_{j+1}}$.
        Furthermore assume that
        $\abs{\lambda_1} > \abs{\lambda_2} > \abs{\lambda_3}$.
        Consider the artificial version of the power method
        $\v{v}^{(k)} = A\v{v}^{(k-1)}/\lambda_1$ with
        $\v{v}^{(0)} = \alpha_1 \v{q}_1 + \cdots + \alpha_m \v{q}_m$, where
        $\alpha_1$ and $\alpha_2$ are both nonzero.
        Show that the sequence converges linearly to $\alpha_1 \v{q}_1$ with
        asymptotic constant $C = \abs{\lambda_2/\lambda_1}$.

        \begin{proof}
            
        \end{proof}

    \item % #4
        Consider the matrix
        \[
            A =
            \begin{bmatrix}
                -1 &  0 &  1 \\
                 1 & -1 &  0 \\
                 0 &  1 & -1 \\
                 1 &  0 &  1
            \end{bmatrix}
        \]
        \begin{enumerate}
            \item[(a)]
                Calculate the eigenvalues and eigenvectors of $A^T A$

                First we must compute the matrix, $A^T A$.
                \[
                    A^T A =
                    \begin{bmatrix}
                        3 & -1 & 0 \\
                        -1 & 2 & -1 \\
                        0 & -1 & 3
                    \end{bmatrix}
                \]
                The eigenvalues can be found by using the characteristic
                polynomial, that is $p(z) = \det(zI - A^T A)$.
                \begin{align*}
                    \det(zI - A^T A) &=
                    \begin{vmatrix}
                        z - 3 &    1 &     0 \\
                           1 & z - 2 &    1 \\
                            0 &    1 & z - 3
                    \end{vmatrix} \\
                    &= (z - 3)^2(z - 2) - (z - 3) - (z - 3) \\
                    &= (z - 3)((z - 3)(z - 2) - 2) \\
                    &= (z - 3)\p{z^2 - 5z + 4} \\
                    &= (z - 3)(z - 4)(z - 1)
                \end{align*}
                The eigenvalues are the zeros of the characteristic polynomial,
                therefore $\spec(A) = \set{1, 3, 4}$.

                The eigenvectors of $A^T A$ can be found by solving the following systems
                \begin{align*}
                    (I - A^T A)\v{x} &= \v{0} \\
                    (3I - A^T A)\v{x} &= \v{0} \\
                    (4I - A^T A)\v{x} &= \v{0} \\
                \end{align*}

                First I will solve $(I - A^T A)\v{x} = \v{0}$ using the
                augmented system.
                \begin{align*}
                    \begin{bmatrix}
                        -2 &  1 &  0 & 0 \\
                         1 & -1 &  1 & 0 \\
                         0 &  1 & -2 & 0
                    \end{bmatrix} \\
                    \begin{bmatrix}
                         1 & -1/2 &  0 & 0 \\
                         1 & -1 &  1 & 0 \\
                         0 &  1 & -2 & 0
                    \end{bmatrix} \\
                    \begin{bmatrix}
                         1 & -1/2 &  0 & 0 \\
                         0 & -1/2 &  1 & 0 \\
                         0 &  1 & -2 & 0
                    \end{bmatrix} \\
                    \begin{bmatrix}
                         1 & -1/2 & 0 & 0 \\
                         0 & 1 & -2 & 0 \\
                         0 & 1 & -2 & 0
                    \end{bmatrix} \\
                    \begin{bmatrix}
                         1 & -1/2 &  0 & 0 \\
                         0 & 1 & -2 & 0 \\
                         0 & 0 & 0 & 0
                    \end{bmatrix} \\
                    \begin{bmatrix}
                         1 & 0 & -1 & 0 \\
                         0 & 1 & -2 & 0 \\
                         0 & 0 &  0 & 0
                    \end{bmatrix} \\
                \end{align*}
                Thus the solutions to this system are of the form
                \begin{align*}
                    \begin{bmatrix}
                        x \\
                        2x \\
                        x
                    \end{bmatrix}
                \end{align*}
                The eigenvector with 2-norm equal to one for eigenvalue 1 is
                \begin{align*}
                    \begin{bmatrix}
                        1/\sqrt{6} \\
                        2/\sqrt{6} \\
                        1/\sqrt{6}
                    \end{bmatrix}
                \end{align*}

                The eigenvector vector for eigenvalue 3 can be found as
                \begin{align*}
                    \begin{bmatrix}
                        0 & 1 & 0 & 0 \\
                        1 & 1 & 1 & 0 \\
                        0 & 1 & 0 & 0
                    \end{bmatrix} \\
                    \begin{bmatrix}
                        1 & 1 & 1 & 0 \\
                        0 & 1 & 0 & 0 \\
                        0 & 1 & 0 & 0
                    \end{bmatrix} \\
                    \begin{bmatrix}
                        1 & 1 & 1 & 0 \\
                        0 & 1 & 0 & 0 \\
                        0 & 0 & 0 & 0
                    \end{bmatrix} \\
                    \begin{bmatrix}
                        1 & 0 & 1 & 0 \\
                        0 & 1 & 0 & 0 \\
                        0 & 0 & 0 & 0
                    \end{bmatrix} \\
                \end{align*}
                Thus the solutions to this system are of the form
                \begin{align*}
                    \begin{bmatrix}
                        x \\
                        0 \\
                        -x
                    \end{bmatrix}
                \end{align*}
                The eigenvector with 2-norm equal to one for eigenvalue 3 is
                \begin{align*}
                    \begin{bmatrix}
                        1/\sqrt{2} \\
                        0 \\
                        -1/\sqrt{2}
                    \end{bmatrix}
                \end{align*}

                Lastly the eigenvector for eigenvalue 4 is needed.
                \begin{align*}
                    \begin{bmatrix}
                        1 & 1 & 0 & 0 \\
                        1 & 2 & 1 & 0 \\
                        0 & 1 & 1 & 0
                    \end{bmatrix} \\
                    \begin{bmatrix}
                        1 & 1 & 0 & 0 \\
                        0 & 1 & 1 & 0 \\
                        0 & 1 & 1 & 0
                    \end{bmatrix} \\
                    \begin{bmatrix}
                        1 & 1 & 0 & 0 \\
                        0 & 1 & 1 & 0 \\
                        0 & 0 & 0 & 0
                    \end{bmatrix} \\
                    \begin{bmatrix}
                        1 & 0 & -1 & 0 \\
                        0 & 1 & 1 & 0 \\
                        0 & 0 & 0 & 0
                    \end{bmatrix} \\
                \end{align*}
                Thus the solutions to this system are of the form
                \begin{align*}
                    \begin{bmatrix}
                        x \\
                        -x \\
                        x
                    \end{bmatrix}
                \end{align*}
                The eigenvector with 2-norm equal to one for eigenvalue 3 is
                \begin{align*}
                    \begin{bmatrix}
                        1/\sqrt{3} \\
                        -1/\sqrt{3} \\
                        1/\sqrt{3}
                    \end{bmatrix}
                \end{align*}

                Thus the eigenvalue decomposition of $A^T A$ is
                \begin{align*}
                    A^T A &= X \Lambda X' \\
                    X &=
                    \begin{bmatrix}
                        1/\sqrt{6} & 1/\sqrt{2}  & 1/\sqrt{3} \\
                        2/\sqrt{6} & 0           & -1/\sqrt{3} \\
                        1/\sqrt{6} & -1/\sqrt{2} & 1/\sqrt{3}
                    \end{bmatrix} \\
                    \Lambda &=
                    \begin{bmatrix}
                        1 & 0 & 0 \\
                        0 & 3 & 0 \\
                        0 & 0 & 4
                    \end{bmatrix}
                \end{align*}

            \item[(b)]
                Use your results in (a) to compute (by hand) the SVD of $A$.

                The singular values of $A$ are the nonnegative square roots of
                the eigenvalues of $A^T A$.
                Thus if $A = U \Sigma V^T$ is a singular value decomposition of
                $A$, then
                \begin{align*}
                    \Sigma &=
                    \begin{bmatrix}
                        1 & 0 & 0 \\
                        0 & \sqrt{3} & 0 \\
                        0 & 0 & 2
                    \end{bmatrix} \\
                    V &=
                    \begin{bmatrix}
                        1/\sqrt{6} & 1/\sqrt{2}  & 1/\sqrt{3} \\
                        2/\sqrt{6} & 0           & -1/\sqrt{3} \\
                        1/\sqrt{6} & -1/\sqrt{2} & 1/\sqrt{3}
                    \end{bmatrix} \\
                \end{align*}
                The unitary matrix $U$ can be found by doing
                Gram-Schmidt on the columns of $AV$.

                \begin{align*}
                    AV &=
                    \begin{bmatrix}
                        0 & -2/\sqrt{2} & 0 \\
                        -1/\sqrt{6} & 1/\sqrt{2} & 2/\sqrt{3} \\
                        1/\sqrt{6} & 1/\sqrt{2} & -2/\sqrt{3} \\
                        2/\sqrt{6} & 0 & 2/\sqrt{3}
                    \end{bmatrix} \\
                    \v{u}_1 &= 
                    \begin{bmatrix}
                        0 \\
                        -1/\sqrt{6} \\
                        1/\sqrt{6} \\
                        2/\sqrt{6}
                    \end{bmatrix} \\
                    \v{u}_2 &= A\v{v}_2
                    - \v{u}_1^T A\v{v}_2 \v{u}_1 \\
                    \v{u}_2 &=
                    \begin{bmatrix}
                        0
                    \end{bmatrix}
                \end{align*}
                
                

            \item[(c)]
                Find the $1$-, $2$-, $\infty$-, and Frobenius norms of $A$.
        \end{enumerate}

    \item % #5
        Write a MATLAB function $[v, lam, k] = Pwr(A, v0)$ that uses the method
        of power iteration to compute the largest eigenvalue, ``$lam$'', and a
        corresponding eigenvector $v$ that has length one in the 2-norm.
        The third argument returned, $k$, should be the number of iterations used in
        the computation.
        The input data is a square matrix $A$ and a starting vector $v0$.

    \item % #6

    \item % #7

\end{enumerate}
\end{document}
