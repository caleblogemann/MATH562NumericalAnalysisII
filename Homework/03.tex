\documentclass[11pt]{article}
\usepackage[letterpaper]{geometry}
\usepackage{MATH562}

\begin{document}
\noindent \textbf{\Large{Caleb Logemann \\
MATH 562 Numerical Analysis II \\
Homework 3
}}

%\lstinputlisting[language=Matlab]{H01_23.m}
\begin{enumerate}
    \item % #1
        Determine the relative condition number for the following problem.
        Are there values of $x$ for which the problem is ill-conditioned?
        Justify your answer.
        \[
            f(x) = \frac{1 - e^{-x}}{1 + e^{-x}}
        \]

        Since $f$ is differentiable the relative condition number of $f$ is given by
        $\kappa = \frac{\abs{f'(x)}}{\abs{f(x)}/\abs{x}}$.
        For this problem
        \begin{align*}
            f'(x) &= \frac{\p{1 + e^{-x}}e^{-x} - \p{1 - e^{-x}}\p{-e^{-x}}}{\p{1 + e^{-x}}^2} \\
                  &= \frac{e^{-x} + e^{-2x} + e^{-x} - e^{-2x}}{\p{1 + e^{-x}}^2} \\ 
                  &= \frac{2e^{-x}}{\p{1 + e^{-x}}^2} \\ 
        \end{align*}
        Thus the relative condition number for this problem is
        \begin{align*}
            \kappa &= \frac{\abs{f'(x)}}{\abs{f(x)}/\abs{x}} \\
            &= 
        \end{align*}


    \item % #2
        Determine whether the calculation $f(x, y) = (1 + x)y^2$ is backward
        stable by the alogirithm
        \[
            \tilde{f}(x, y) = \br{1 \oplus fl(x)} \otimes \br{fl(y) \otimes fl(y)}
        \]

    \item % #3
        \begin{enumerate}
            \item[(a)]
                Compute the LU factorization $A = LU$, of
                \[
                    A &=
                    \begin{bmatrix}
                        1 & 2 & 4 \\
                        2 & 3 & 4 \\
                        2 & 5 & 6
                    \end{bmatrix}
                \]
                Use the factorization to solve the system $A\v{x} = \v{b}$
                where $\v{b} = \br{-1 1 1}^T$.

            \item[(b)]
                Solve the system $A\v{x} = \v{b}$ by LU factorization with
                partial pivoting
        \end{enumerate}

    \item % #4
        Let $A \in \CC^{m \times m}$ be nonsingular.
        Show that $A$ has an LU factorization if and only if for each $k$, such
        that $1 \le k \le m$, the upper left $(k \times k)$ block $A(1:k, 1:k)$
        of $A$ is nonsingular.
        Show that this LU factorization is unique.

    \item % #5
        Rank Deficient Least Squares Problem:
    \item % #6
    \item % #7
\end{enumerate}
\end{document}
