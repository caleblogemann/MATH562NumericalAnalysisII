\documentclass[11pt]{article}
\usepackage[letterpaper]{geometry}
\usepackage{MATH562}

\begin{document}
\noindent \textbf{\Large{Caleb Logemann \\
MATH 562 Numerical Analysis II \\
Homework 3
}}

%\lstinputlisting[language=Matlab]{H01_23.m}
\begin{enumerate}
    \item % #1 Done
        Determine the relative condition number for the following problem.
        Are there values of $x$ for which the problem is ill-conditioned?
        Justify your answer.
        \[
            f(x) = \frac{1 - e^{-x}}{1 + e^{-x}}
        \]

        Since $f$ is differentiable the relative condition number of $f$ is given by
        $\kappa = \frac{\abs{f'(x)}}{\abs{f(x)}/\abs{x}}$.
        For this problem
        \begin{align*}
            f'(x) &= \frac{\p{1 + e^{-x}}e^{-x} - \p{1 - e^{-x}}\p{-e^{-x}}}{\p{1 + e^{-x}}^2} \\
                  &= \frac{e^{-x} + e^{-2x} + e^{-x} - e^{-2x}}{\p{1 + e^{-x}}^2} \\ 
                  &= \frac{2e^{-x}}{\p{1 + e^{-x}}^2} \\ 
        \end{align*}
        Thus the relative condition number for this problem is
        \begin{align*}
            \kappa &= \frac{\abs{f'(x)}}{\abs{f(x)}/\abs{x}} \\
                   &= \abs{\frac{2xe^{-x}}{\p{1 + e^{-x}}^2} / \frac{1 - e^{-x}}{1 + e^{-x}}} \\
                   &= \abs{\frac{2xe^{-x}}{\p{1 + e^{-x}}^2} \times \frac{1 + e^{-x}}{1 - e^{-x}}} \\
                   &= \abs{\frac{2xe^{-x}}{\p{1 + e^{-x}}} \times \frac{1}{1 - e^{-x}}} \\
                   &= \abs{\frac{2xe^{-x}}{\p{1 - e^{-2x}}}} \\
        \end{align*}

        This problem is not ill-conditioned because for any $x$ this relitive
        condition number is small.
        At $x = 0$, this condition number is undefined, but L'Hopital's rule
        shows that the limit is equal to $1$.
        \begin{align*}
            \lim{x \to 0}{\kappa} &= \lim{x \to 0}{\frac{2 e^{-x} - 2 x e^{-x}}{2 e^{-2x}}} \\
                                  &= \frac{2 e^{0}}{2 e^{0}} \\
                                  &= 1
        \end{align*}
        As $x \to \infty$, $2xe^{-x} \to 0$ and $1 - e^{-2x} \to 1$, therefore $\kappa \to 0$.
        As $x \to -\infty$, $1 - e^{-2x} > 2xe^{-x}$, so $\kappa \to 0$.
        In fact $\kappa \le 1$ for all $x$, therefore this problem is not
        ill-conditioned.

    \item % #2 Done
        Determine whether the calculation $f(x, y) = (1 + x)y^2$ is backward
        stable by the alogirithm
        \[
            \tilde{f}(x, y) = \br{1 \oplus fl(x)} \otimes \br{fl(y) \otimes fl(y)}
        \]

        The algorithm $\tilde{f}$ is backward stable if there exists
        $\tilde{\v{x}} = (\tilde{x}, \tilde{y})$ such that
        $\tilde{f}(x, y) = f(\tilde{x}, \tilde{y})$ and
        $\frac{\norm{\v{x} - \tilde{\v{x}}}}{\norm{\v{x}}} = O(\epsilon_{machine})$
        for all $\v{x}$.

        \begin{align*}
            \tilde{f}(x, y) &= \br{1 \oplus fl(x)} \otimes \br{fl(y) \otimes fl(y)} \\
            &= \br{1 \oplus x(1 + \epsilon_1)} \otimes \br{y(1 + \epsilon_2) \otimes y(1 + \epsilon_3)} \\
            &= \br{1 + x(1 + \epsilon_1)}(1 + \epsilon_4) \otimes \br{y(1 + \epsilon_2) \times y(1 + \epsilon_3)}(1 + \epsilon_5) \\
            &= \br{1 + x(1 + \epsilon_1)}(1 + \epsilon_4) \times \br{y(1 + \epsilon_2) \times y(1 + \epsilon_3)}(1 + \epsilon_5)(1 + \epsilon_6) \\
            &= \br{1 + x(1 + \epsilon_1)} y^2 (1 + \epsilon_7)
            \intertext{where $\epsilon_7 = O(\epsilon_{machine})$}
            \tilde{f}(x, y) &= \br{1 + x(1 + \epsilon_1)} y^2 (1 + \epsilon_7) \\
                            &= \br{1 + x(1 + \epsilon_1)} \p{y \sqrt{1 + \epsilon_7}}^2 \\
                            &= \br{1 + x(1 + \epsilon_1)} \p{y (1 + \epsilon_8)}^2 \\
                            &= f(x(1 + \epsilon_1), y (1 + \epsilon_8))
        \end{align*}
        Therefore $\tilde{x} = x(1 + \epsilon_1)$ and
        $\tilde{y} = y (1 + \epsilon_8)$.
        This does satisfy $\frac{\norm{\v{x} - \tilde{\v{x}}}}{\norm{\v{x}}} = O(\epsilon_{machine})$,
        because $\norm{\v{x} - \tilde{\v{x}}} = \sqrt{\epsilon_1^2 + \epsilon_8^2} = O(\epsilon_{machine})$.
        So this algorithm is backward stable.

    \item % #3
        \begin{enumerate}
            \item[(a)]
                Compute the LU factorization $A = LU$, of
                \[
                    A =
                    \begin{bmatrix}
                        1 & 2 & 4 \\
                        2 & 3 & 4 \\
                        2 & 5 & 6
                    \end{bmatrix}
                \]
                Use the factorization to solve the system $A\v{x} = \v{b}$
                where $\v{b} = \br{-1 1 1}^T$.


            \item[(b)]
                Solve the system $A\v{x} = \v{b}$ by LU factorization with
                partial pivoting
        \end{enumerate}

    \item % #4
        Let $A \in \CC^{m \times m}$ be nonsingular.
        Show that $A$ has an LU factorization if and only if for each $k$, such
        that $1 \le k \le m$, the upper left $(k \times k)$ block $A(1:k, 1:k)$
        of $A$ is nonsingular.
        Show that this LU factorization is unique.

    \item % #5
        Rank Deficient Least Squares Problem:

    \item % #6
        Consider the matrix
        \[
            A =
            \begin{bmatrix}
                1 & 2 \\
                0 & 1 \\
                1 & 0 \\
            \end{bmatrix}
        \]
        \begin{enumerate}
            \item[(a)]
                Using any method you like, determine reduced and full $QR$
                factorizations.

            \item[(b)]
                Use the $QR$ factorization to solve the linear least square problem
                \[
                    \min*_{\v{x}}\norm[2]{A\v{x} - \v{b}}^2
                \]
                with $\v{b} = \br{1 1 0}^T$.

            \item[(c)]
                Use the $QR$ factorization to solve the linear least squares
                problem
                \[
                    \min*_{\v{x}}\norm[2]{A\v{x} - \v{b}}^2
                \]
                with matrix $A \in \RR^{m \times n}$ with rank $n$ and
                $\v{b} \in \RR^m$.
        \end{enumerate}

    \item % #7


\end{enumerate}
\end{document}
